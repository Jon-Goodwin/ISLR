% Options for packages loaded elsewhere
\PassOptionsToPackage{unicode}{hyperref}
\PassOptionsToPackage{hyphens}{url}
%
\documentclass[
]{article}
\usepackage{amsmath,amssymb}
\usepackage{lmodern}
\usepackage{iftex}
\ifPDFTeX
  \usepackage[T1]{fontenc}
  \usepackage[utf8]{inputenc}
  \usepackage{textcomp} % provide euro and other symbols
\else % if luatex or xetex
  \usepackage{unicode-math}
  \defaultfontfeatures{Scale=MatchLowercase}
  \defaultfontfeatures[\rmfamily]{Ligatures=TeX,Scale=1}
\fi
% Use upquote if available, for straight quotes in verbatim environments
\IfFileExists{upquote.sty}{\usepackage{upquote}}{}
\IfFileExists{microtype.sty}{% use microtype if available
  \usepackage[]{microtype}
  \UseMicrotypeSet[protrusion]{basicmath} % disable protrusion for tt fonts
}{}
\makeatletter
\@ifundefined{KOMAClassName}{% if non-KOMA class
  \IfFileExists{parskip.sty}{%
    \usepackage{parskip}
  }{% else
    \setlength{\parindent}{0pt}
    \setlength{\parskip}{6pt plus 2pt minus 1pt}}
}{% if KOMA class
  \KOMAoptions{parskip=half}}
\makeatother
\usepackage{xcolor}
\usepackage[margin=1in]{geometry}
\usepackage{color}
\usepackage{fancyvrb}
\newcommand{\VerbBar}{|}
\newcommand{\VERB}{\Verb[commandchars=\\\{\}]}
\DefineVerbatimEnvironment{Highlighting}{Verbatim}{commandchars=\\\{\}}
% Add ',fontsize=\small' for more characters per line
\usepackage{framed}
\definecolor{shadecolor}{RGB}{248,248,248}
\newenvironment{Shaded}{\begin{snugshade}}{\end{snugshade}}
\newcommand{\AlertTok}[1]{\textcolor[rgb]{0.94,0.16,0.16}{#1}}
\newcommand{\AnnotationTok}[1]{\textcolor[rgb]{0.56,0.35,0.01}{\textbf{\textit{#1}}}}
\newcommand{\AttributeTok}[1]{\textcolor[rgb]{0.77,0.63,0.00}{#1}}
\newcommand{\BaseNTok}[1]{\textcolor[rgb]{0.00,0.00,0.81}{#1}}
\newcommand{\BuiltInTok}[1]{#1}
\newcommand{\CharTok}[1]{\textcolor[rgb]{0.31,0.60,0.02}{#1}}
\newcommand{\CommentTok}[1]{\textcolor[rgb]{0.56,0.35,0.01}{\textit{#1}}}
\newcommand{\CommentVarTok}[1]{\textcolor[rgb]{0.56,0.35,0.01}{\textbf{\textit{#1}}}}
\newcommand{\ConstantTok}[1]{\textcolor[rgb]{0.00,0.00,0.00}{#1}}
\newcommand{\ControlFlowTok}[1]{\textcolor[rgb]{0.13,0.29,0.53}{\textbf{#1}}}
\newcommand{\DataTypeTok}[1]{\textcolor[rgb]{0.13,0.29,0.53}{#1}}
\newcommand{\DecValTok}[1]{\textcolor[rgb]{0.00,0.00,0.81}{#1}}
\newcommand{\DocumentationTok}[1]{\textcolor[rgb]{0.56,0.35,0.01}{\textbf{\textit{#1}}}}
\newcommand{\ErrorTok}[1]{\textcolor[rgb]{0.64,0.00,0.00}{\textbf{#1}}}
\newcommand{\ExtensionTok}[1]{#1}
\newcommand{\FloatTok}[1]{\textcolor[rgb]{0.00,0.00,0.81}{#1}}
\newcommand{\FunctionTok}[1]{\textcolor[rgb]{0.00,0.00,0.00}{#1}}
\newcommand{\ImportTok}[1]{#1}
\newcommand{\InformationTok}[1]{\textcolor[rgb]{0.56,0.35,0.01}{\textbf{\textit{#1}}}}
\newcommand{\KeywordTok}[1]{\textcolor[rgb]{0.13,0.29,0.53}{\textbf{#1}}}
\newcommand{\NormalTok}[1]{#1}
\newcommand{\OperatorTok}[1]{\textcolor[rgb]{0.81,0.36,0.00}{\textbf{#1}}}
\newcommand{\OtherTok}[1]{\textcolor[rgb]{0.56,0.35,0.01}{#1}}
\newcommand{\PreprocessorTok}[1]{\textcolor[rgb]{0.56,0.35,0.01}{\textit{#1}}}
\newcommand{\RegionMarkerTok}[1]{#1}
\newcommand{\SpecialCharTok}[1]{\textcolor[rgb]{0.00,0.00,0.00}{#1}}
\newcommand{\SpecialStringTok}[1]{\textcolor[rgb]{0.31,0.60,0.02}{#1}}
\newcommand{\StringTok}[1]{\textcolor[rgb]{0.31,0.60,0.02}{#1}}
\newcommand{\VariableTok}[1]{\textcolor[rgb]{0.00,0.00,0.00}{#1}}
\newcommand{\VerbatimStringTok}[1]{\textcolor[rgb]{0.31,0.60,0.02}{#1}}
\newcommand{\WarningTok}[1]{\textcolor[rgb]{0.56,0.35,0.01}{\textbf{\textit{#1}}}}
\usepackage{graphicx}
\makeatletter
\def\maxwidth{\ifdim\Gin@nat@width>\linewidth\linewidth\else\Gin@nat@width\fi}
\def\maxheight{\ifdim\Gin@nat@height>\textheight\textheight\else\Gin@nat@height\fi}
\makeatother
% Scale images if necessary, so that they will not overflow the page
% margins by default, and it is still possible to overwrite the defaults
% using explicit options in \includegraphics[width, height, ...]{}
\setkeys{Gin}{width=\maxwidth,height=\maxheight,keepaspectratio}
% Set default figure placement to htbp
\makeatletter
\def\fps@figure{htbp}
\makeatother
\setlength{\emergencystretch}{3em} % prevent overfull lines
\providecommand{\tightlist}{%
  \setlength{\itemsep}{0pt}\setlength{\parskip}{0pt}}
\setcounter{secnumdepth}{-\maxdimen} % remove section numbering
\ifLuaTeX
  \usepackage{selnolig}  % disable illegal ligatures
\fi
\IfFileExists{bookmark.sty}{\usepackage{bookmark}}{\usepackage{hyperref}}
\IfFileExists{xurl.sty}{\usepackage{xurl}}{} % add URL line breaks if available
\urlstyle{same} % disable monospaced font for URLs
\hypersetup{
  pdftitle={Chapter 3 Linear Regression},
  hidelinks,
  pdfcreator={LaTeX via pandoc}}

\title{Chapter 3 Linear Regression}
\author{}
\date{\vspace{-2.5em}}

\begin{document}
\maketitle

\begin{Shaded}
\begin{Highlighting}[]
\FunctionTok{library}\NormalTok{(tidyverse)}
\end{Highlighting}
\end{Shaded}

\begin{verbatim}
## -- Attaching core tidyverse packages ------------------------ tidyverse 2.0.0 --
## v dplyr     1.1.0     v readr     2.1.4
## v forcats   1.0.0     v stringr   1.5.0
## v ggplot2   3.4.1     v tibble    3.2.0
## v lubridate 1.9.2     v tidyr     1.3.0
## v purrr     1.0.1     
## -- Conflicts ------------------------------------------ tidyverse_conflicts() --
## x dplyr::filter() masks stats::filter()
## x dplyr::lag()    masks stats::lag()
## i Use the ]8;;http://conflicted.r-lib.org/conflicted package]8;; to force all conflicts to become errors
\end{verbatim}

\begin{Shaded}
\begin{Highlighting}[]
\FunctionTok{library}\NormalTok{(ISLR2)}
\end{Highlighting}
\end{Shaded}

\hypertarget{simple-linear-regression}{%
\subsubsection{Simple Linear
Regression}\label{simple-linear-regression}}

Mathematically a linear relation is the form
\(Y \sim \beta_0 + \beta_1 X\) (3.1)

Where \(\beta_0\) is the intercept coefficient and \(\beta_1\) the
slope.

\begin{itemize}
\tightlist
\item
  Typically referred to regressing \(Y\) on \(X\)
\end{itemize}

Once the training data produces estimates for \(\hat\beta_0\) and
\(\hat\beta_1\) We can comput \(\hat{y} = \hat\beta_0 + \hat\beta_1 x\)
(3.2)

\hypertarget{estimating-the-coefficients}{%
\paragraph{Estimating the
Coefficients}\label{estimating-the-coefficients}}

\(e_i = y_i-\hat{y}_i\) represents the ith residual.

We define the residual sum of squares (RSS) as
\(RSS=e_{1}^2 + e_{2}^2 +\cdot\cdot\cdot+e_{n}^2\)

the minimizers are then:

\(\hat\beta_1 = \frac{\sum_{i=1}^n (x_i-\bar{x})(y_i-\bar{y})}{\sum_i={1}^n(x_i-\bar{x})^2}\)

\(\hat\beta_0 = \bar{y} - \hat\beta_1 \bar{x}\)

\hypertarget{assessing-the-accuracy-of-the-coefficient-estiamtes}{%
\paragraph{Assessing the Accuracy of the Coefficient
Estiamtes}\label{assessing-the-accuracy-of-the-coefficient-estiamtes}}

The well known standard error formula

\(Var(\hat{\mu}) = SE(\hat{\mu})^2 = \frac{\sigma^2}{n}\) (3.7)

extends to \(\hat\beta_0,\hat\beta_1\):

\(SE(\hat\beta_0)^2 = \sigma^2\big(\frac{1}{n} + \frac{\bar{x}^2}{\sum_{i=1}^n(x_i-\bar{x})^2}\big)\)

\(SE(\hat\beta_1)^2 = \frac{\sigma^2}{\sum_{i=1}^n (x_i-\bar{x})^2}\)
(3.8)

where \(\sigma^2 = Var(\epsilon)\)

The estimate of \(\sigma\) is known as the residual standard error given
by:

\(RSE = \sqrt{RSS/(n-2)}\)

Confidence Intervals:

\(\hat\beta_1 \pm 2\cdot SE(\hat\beta_1)\)

\(\hat\beta_0 \pm 2\cdot SE(\hat\beta_0)\)

Hypothesis Test

\(H_0 :\beta_1 = 0\) vs \(H_a :\beta_1 \neq 0\)

Test statistics \(t = \frac{\hat\beta_1 - 0}{SE(\hat\beta_1)}\)

Residual Standard Error

\(RSE = \sqrt{\frac{1}{n-2} RSS} = \sqrt{\frac{1}{n-2} \sum_{i=1}^n (y_i-\hat y_i)^2}\)

\(R^2\) Statistic.

\(R^2 = \frac{TSS - RSS}{TSS} = 1-\frac{RSS}{TSS}\) where
\(TSS = \sum(y_i-\bar{y})^2\) is the total sum of squares.

TSS measures the total variancei n the response Y.

Recall correlation formula, it can be shown that Cor(X,Y) = r and in
linear regression that \(R^2 = r^2\)

\hypertarget{multiple-linear-regression}{%
\subsubsection{Multiple Linear
Regression}\label{multiple-linear-regression}}

For \(p\) predictors the multiple linear regression model takes the
form:

\(Y = \beta_0+\beta_1 X_1 + \beta_2 X_2 + \cdot\cdot\cdot + \beta_p X_p +\epsilon\)

We interpret each \(\beta_j\) as the average effect on Y of a one unit
increase in \(X_j\) holding all other predictors fixed.

\hypertarget{estimating-the-regression-coefficients}{%
\paragraph{Estimating the Regression
Coefficients}\label{estimating-the-regression-coefficients}}

The general form of RSS is the same
\(RSS = \sum_{i=1}^n(y_i-\hat y_i)^2\) though
\(\hat y_i = \hat\beta_0+\hat\beta_1 x_1 +...+\hat\beta_p x_p\)

\hypertarget{some-important-questions}{%
\paragraph{Some Important Questions}\label{some-important-questions}}

\begin{enumerate}
\def\labelenumi{\arabic{enumi}.}
\item
  Is at least one of the predictors \(X_1,...,X_p\) useful in predicting
  the response?
\item
  Do all the predictors help to explain Y, or is only a subset of the
  predictors useful?
\item
  How well does the model fit the data?
\item
  Given a set of predictors values, what response value should we
  predict and how accurate is our prediction.
\end{enumerate}

\hypertarget{is-at-least-one-of-the-predictors-x_1...x_p-useful-in-predicting-the-response}{%
\paragraph{\texorpdfstring{Is at least one of the predictors
\(X_1,...,X_p\) useful in predicting the
response?}{Is at least one of the predictors X\_1,...,X\_p useful in predicting the response?}}\label{is-at-least-one-of-the-predictors-x_1...x_p-useful-in-predicting-the-response}}

Hypothesis test:

\(H_0: \beta_1=\beta_2...=\beta_p=0\) vs \(H_a:\) at least one of
\(\beta_j\) is non-zero.

Using test statistic \(F= \frac{TTSS-RSS)/p}{RSS/(n-p-1)}\), TSS and RSS
are the same as simple linear regression.

One can show that \(E(RSS/(n-p-1)) = \sigma^2\) and that provided
\(H_0\) is true \(E((TSS-RSS)/p) = \sigma^2\)

To test that some subset q of the coefficients are zero the null is:

\$ H\_0: \beta\emph{\{p-q+1\} = \beta}\{p-q+2\}=\ldots=\beta\_0 = 0\$

In this case we fit a second model that uses all the variables except
those \(q\)

Suppose \(RSS_0\) is the residual sum of squares for such a model then:

\(F = \frac{RSS_0 - RSS)/q}{RSS/(n-p-1)}\)

When the number of predictors is high , the multiple linear regression
model cannot be fit using least squares and so the F-statistic cannot be
used.

\hypertarget{two-deciding-on-important-variables}{%
\paragraph{Two: Deciding on Important
Variables}\label{two-deciding-on-important-variables}}

Three approaches: Forward Selection, Backward Selection, Mixed Selection

\hypertarget{three-model-fit}{%
\paragraph{Three: Model Fit}\label{three-model-fit}}

Adding predictors reduces RSS and thus increases \(R^2\). It may or may
not reduce RSE depending on whether the reduction in RSS outweighs the
increase in p

\hypertarget{four-predictions}{%
\paragraph{Four: Predictions}\label{four-predictions}}

Three sorts of uncertainty associated with rpedicition:

\begin{enumerate}
\def\labelenumi{\arabic{enumi}.}
\tightlist
\item
  The coefficient estiamte \(\hat\beta_0,...,\hat\beta_p\) are estimates
  for \(\beta_0,...,\beta_p\) That is, the least squares plane
  \(\hat Y = \hat\beta_0 +\beta_1 X_1 +\cdot\cdot\cdot+\hat\beta_p X_p\)
\end{enumerate}

is only an estimate for the true population regression plane
\(f(X) = \beta+\beta_1 X_1 +\cdot\cdot\cdot+ \beta_p X_p\)

We can compute a confidence interval in order to determine how close
\(\hat Y\) will be to \(f(X)\)

\begin{enumerate}
\def\labelenumi{\arabic{enumi}.}
\setcounter{enumi}{1}
\item
  Of course, in practice assuming a linear model for \(f(X)\) is almost
  always an approximation of reality so there is an additional source of
  potentiall reducible error which we call model bias.
\item
  Even if we knew the true values we can't predict the response due to
  \(\epsilon\)
\end{enumerate}

\hypertarget{other-considerations-in-the-regression-model}{%
\subsection{Other Considerations in the Regression
Model}\label{other-considerations-in-the-regression-model}}

\hypertarget{qualitative-predictors}{%
\paragraph{Qualitative Predictors}\label{qualitative-predictors}}

Qualitative predictors can be encoded as dummy variables \(x_i\)

where \(x_i = 1\) if ith person has the quality and \(x_i = 0\) if ith
person does not have the quality.

There will always be one fewer dummy variable then the number of
factors.

\hypertarget{removing-the-additive-assumption}{%
\paragraph{Removing the additive
assumption}\label{removing-the-additive-assumption}}

The hierarchical principle states that if we include an interaction in a
model, we hierarchical should also include the main effects, even if the
p-values associated with principle their coefficients are not
significant.

\hypertarget{potention-problems}{%
\paragraph{3.3.3 Potention problems}\label{potention-problems}}

\begin{enumerate}
\def\labelenumi{\arabic{enumi}.}
\item
  Non-linearity of the response-predictor relationships.
\item
  Correlation of error terms.
\item
  Non-constant variance of error terms.
\item
  Outliers.
\item
  High-leverage points.
\item
  Collinearity.
\end{enumerate}

Leverage Statistics:
\(h_=\frac{1}{n} +\frac{(x_i-\bar{x}^2)}{\sum_{i'=1}^n(x_{i'}-\bar{x})^2}\)

\end{document}
